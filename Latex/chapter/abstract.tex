\clearpage
{\normalfont
\color{uniblau}
\huge\sffamily\itshape
Kurzzusammenfassung
}

Im traditionellen Sinne bezieht sich der Begriff Typografie auf die Gestaltung von Druckwerken mit beweglichen Lettern (Typen). Anfänglich fand dies insbesondere im Bleisatz bzw. dem Satz mit Holzlettern statt.

In der Medientheorie steht Typografie für gedruckte Schrift in Abgrenzung zu Handschrift (Chirografie) und elektronischen sowie nicht literalen Texten.

Heute bezeichnet Typografie meist den medienunabhängigen Gestaltungsprozess, der mittels Schrift, Bildern, Linien, Flächen und Leerräumen alle Arten von Kommunikationsmedien gestaltet. Typografie ist in Abgrenzung zu Kalligrafie, Schreiben oder Schriftentwurf das Gestalten mit vorgefundenem Material.

\vfill

\rule{\textwidth}{0.4pt}

\vspace{1cm}

{\normalfont
\color{uniblau}
\huge\sffamily\itshape
Abstract
}

In the traditional sense, the term typography refers to the design of printed works with movable letters (types). Initially, this was done in lead typesetting or wood typesetting.

In media theory, typography stands for printed type in contrast to handwriting (chirography) and electronic as well as non-literal texts.

Today, typography usually refers to the media-independent design process that uses type, images, lines, surfaces and empty spaces to create all kinds of communication media. In contrast to calligraphy, writing or type design, typography is the design with found material.

% Translated with www.DeepL.com/Translator

\vfill